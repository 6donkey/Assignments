\documentclass[journal,12pt,twocolumn]{article}\usepackage[margin=1.25 in]{geometry}
\usepackage{amsmath, amssymb}
\usepackage{multicol}
\usepackage{graphicx}
\usepackage[shortlabels]{enumitem}

\providecommand{\pr}[1]{\ensuremath{\text{P}\left(#1\right)}}

\addtolength{\topmargin}{-.4 in}

\title{\LARGE{\textbf{Assignment-3}\\(CBSE 12th Ex 23)}}
\author{\normalsize J Sai Sri Hari Vamshi\\ \footnotesize AI21BTECH11014}
\date{}

\begin{document}
\maketitle
\begin{center}
    \textbf{\large Problem 13:}
\end{center}
\noindent Fill in the blanks in the following Table 1:

\begin{table}[h!]
\label{table}
\begin{tabular}{|c|c|c|c|} 
\hline
\textbf{$\pr A$} & \textbf{$\pr B$} & \textbf{$\pr{A \cap B}$} & \textbf{$\pr{A \cup B}$} \\
\hline
$\frac{1}{3}$ & $\frac{1}{5}$ & $\frac{1}{15}$ & $...$ \\
\hline
$0.35$ & $...$ & $0.25$ & $0.6$ \\
\hline
$0.5$ & $0.35$ & $...$ & $0.7$ \\
\hline
\end{tabular}
\caption{}
\end{table}

\begin{center}
    \textbf{\large Solution:}
\end{center}

\noindent The given inputs of probabilities involving events $A$ and $B$ vary among $\pr A$, $\pr B$, $\pr{AB}$ and $\pr{A+B}$ with one of them being an unknown.
\noindent We know, if events $E_1$ and $E_2$ were disjoint, we get
\begin{align}
\pr{E_1 + E_2} & = \pr{E_1} + \pr{E_2} \\
\pr{E_1 E_2} & = 0
\end{align}
\noindent Now, for events $A$ and $B$,
\begin{align*}
A + B & = A + (B - A) \\
B - A & = B - AB
\end{align*}
\noindent Here, $A$ and $B - A$ are disjoint. And from equation (1)
\begin{align*}
\pr{A + B} & = \pr{A + (B - A)}\\
& = \pr A + \pr{B - A}\\
& = \pr A + \pr{B - AB}
\end{align*}
\noindent $B$ and $AB$ are also disjoint, so again using equation (1)
\begin{align}
\pr{A + B} & = \pr A + \pr B - \pr{AB}
\end{align}
\noindent This is the General Probabiity Addition Rule.\\
\noindent Using the above formulae (3) we can find the unknown probabilities in the question,
\begin{enumerate}[(a)]
    \item Given values from the question are in Table 2:\\
    	\begin{table}[h!]
			\label{table}
			\begin{tabular}{|c|c|c|} 
\hline
\textbf{Event} & \textbf{Probability} & \textbf{Value} \\
\hline
$A$ & $\pr A$ & $\frac{1}{3}$ \\
\hline
$B$ & $\pr B$ & $\frac{1}{5}$ \\
\hline
$AB$ & $\pr{AB}$ & $\frac{1}{15}$ \\
\hline
$A + B$ & $\pr{A + B}$ & $?$ \\
\hline
\end{tabular}
			\caption{}
		\end{table}
		
		\noindent Substituting the values above into the formulae (3),
		\begin{align*}
		\pr{A + B} & = \pr A + \pr B - \pr{AB} \\
		& = \frac{1}{3} + \frac{1}{5} - \frac{1}{15} \\
		& = \frac{7}{15}
		\end{align*}
		Therefore from above, we get the desired probability,
		\begin{align*}
		\pr{A + B} & = \frac{7}{15}
		\end{align*}
	\item Given values from the question are in Table 3:\\
		\begin{table}[h!]
			\label{table}
			\begin{tabular}{|c|c|c|} 
\hline
\textbf{Event} & \textbf{Probability} & \textbf{Value} \\
\hline
$A$ & $\pr A$ & $0.35$ \\
\hline
$B$ & $\pr B$ & $?$ \\
\hline
$AB$ & $\pr{AB}$ & $0.25$ \\
\hline
$A + B$ & $\pr{A + B}$ & $0.6$ \\
\hline
\end{tabular}
			\caption{}
		\end{table}
		
		\noindent Substituting the values above into the formulae (3),
		\begin{align*}
		\pr{A + B} & = \pr A + \pr B - \pr{AB} \\
		\pr B & = \pr{A + B} + \pr{AB} - \pr B \\
		& = 0.6 + 0.25 - 0.35 \\
		& = 0.5
		\end{align*}
		Therefore from above, we get the desired probability,
		\begin{align*}
		\pr{B} & = 0.5
		\end{align*}
	\item Given values from the question are in Table 4:\\
		\begin{table}[h!]
			\label{table}
			\begin{tabular}{|c|c|c|} 
\hline
\textbf{Event} & \textbf{Probability} & \textbf{Value} \\
\hline
$A$ & $\pr A$ & $0.5$ \\
\hline
$B$ & $\pr B$ & $0.35$ \\
\hline
$AB$ & $\pr{AB}$ & $?$ \\
\hline
$A + B$ & $\pr{A + B}$ & $0.7$ \\
\hline
\end{tabular}
			\caption{}
		\end{table}
		
		\noindent Substituting the values above into the formulae (3),
		\begin{align*}
		\pr{A + B} & = \pr A + \pr B - \pr{AB} \\
		\pr{AB} & = \pr{A} + \pr{B} - \pr{A + B} \\
		& = 0.5 + 0.35 - 0.7 \\
		& = 0.15
		\end{align*}
		Therefore from above, we get the desired probability,
		\begin{align*}
		\pr{AB} & = 0.15
		\end{align*}
\end{enumerate}
\end{document}