\documentclass[journal,12pt,twocolumn]{article}\usepackage[margin=1.25 in]{geometry}
\usepackage{amsmath, amssymb}
\usepackage{multicol}
\usepackage{graphicx}

\providecommand{\pr}[1]{\ensuremath{\text{P}\left(#1\right)}}

\addtolength{\topmargin}{-.4 in}

\title{\LARGE{\textbf{Assignment-2}\\(ICSE 12, 2019)}}
\author{\normalsize J Sai Sri Hari Vamshi\\ \footnotesize AI21BTECH11014}
\date{}

\begin{document}
\maketitle
\begin{center}
    \textbf{\large Problem 1(x):}
\end{center}
\noindent If events $A$ and $B$ are independent such that $\pr{A} = \frac{3}{5}$, $\pr{B} = \frac{2}{3}$, find $\pr{A + B}$.

\begin{center}
    \textbf{\large Solution:}
\end{center}

\noindent The given input probabilities and desired values are given in the Table 1,
\begin{center}
\begin{table}[h!]
\label{table:table1}
\begin{tabular}{|c|c|c|} 
\hline
\textbf{Event} & \textbf{Probability} & \textbf{Value} \\
\hline
$A$ & $\pr A$ & $\frac{3}{5}$ \\
\hline
$B$ & $\pr B$ & $\frac{2}{3}$ \\
\hline
$A \cup B$ & $\pr{A \cup B}$ & $?$ \\
\hline
\end{tabular}
\caption{}
\end{table}
\end{center}
\noindent It is also given that events $A$ and $B$ are independent which means,
\begin{align}
\pr{AB} & = \pr A \pr B
\end{align}
\noindent We know, if events $E_1$ and $E_2$ were disjoint, we get
\begin{align}
\pr{E_1 + E_2} & = \pr{E_1} + \pr{E_2}
\end{align}
\noindent Now, for events $A$ and $B$,
\begin{align*}
A + B & = A + (B - A)
\end{align*}
\noindent Here, $A$ and $B - A$ are disjoint. And from equation (2)
\begin{align*}
\pr{A + B} & = \pr{A + (B - A)}\\
& = \pr A + \pr{B - A}\\
& = \pr A + \pr{B - AB}
\end{align*}
\noindent $B$ and $AB$ are also disjoint, so again using (2)
\begin{align}
\pr{A + B} & = \pr A + \pr B - \pr{AB}
\end{align}
\noindent This is the General Probabiity Addition Rule.\\
\noindent Using the above two formulas, (1) and (3), we can use the modified probability addition rule for independent sets as,
\begin{align}
\pr{A + B} & = \pr A + \pr B - \pr A \pr B
\end{align}
\noindent By substituting the respective values in (4), we get,
\begin{align*}
\pr{A + B} & = \frac{3}{5} + \frac{2}{3} - \frac{3}{5} \cdot \frac{2}{3} \\
& = \frac{3}{5} + \frac{2}{3} - \frac{2}{5}\\
& = \frac{13}{15}
\end{align*}
\noindent So, the desired probability $\pr{A + B}$ is found to be,
\begin{align*}
\pr{A + B} & = \frac{13}{15} = 0.8667
\end{align*}

\end{document}