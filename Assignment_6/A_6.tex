\documentclass{beamer}
\usepackage{changepage}


\providecommand{\x}[1]{\ensuremath{\text{x}\left(#1\right)}}
\providecommand{\p}[1]{\ensuremath{{P_X}\left(#1\right)}}
\providecommand{\f}[1]{\ensuremath{{F_X}\left(#1\right)}}

% Theme choice:
\usetheme{Singapore}

% Title page details: 
\title{Assignment-6\\(Papoulis Chapter 4 Example 4.2)} 
\author{J Sai Sri Hari Vamshi\\ AI21BTECH11014}
\date{\today}
\logo{\large \LaTeX{}}


\begin{document}

% Title page frame
\begin{frame}
    \titlepage 
\end{frame}

% Remove logo from the next slides
%\logo{}


% Outline frame
\begin{frame}{Contents}
    \tableofcontents
\end{frame}

\section{Question}
\begin{frame}{Question}
	In a die roll experiment the six outcomes are denoted as $f_i$ and for each outcome, a random variable is assigned as $\x{f_i} = 10i$.\\ Using these find the set of favourable outcomes for each of the conditions givn below.
	\begin{enumerate}[(i)]
	\item $\{x \leq 35\}$
	\item $\{x \leq 5\}$
	\item $\{20 \leq x \leq 35\}$
	\item $\{x = 40\}$
	\item $\{x = 35\}$
	\end{enumerate}
\end{frame}

\section{Solution}

\begin{frame}{Solution}
Given that the random variables of the outcomes of a die roll experiment are given as $\x{f_i} = 10i$, where $f_i,\ i = 1,2,...6$ are the outcomes.\\ Then for each of the conditions imposed for the random variabes, the favourable outcomes set:
\end{frame}

\begin{frame}{Solution}
\begin{enumerate}
\item[(i)] Given condition,
\begin{align*}
\{x \leq 35\}
\end{align*}
This can be written as,
\begin{align*}
\x{f_i} & \leq 35\\
10i & \leq 35\\
i & \leq 3.5\\
\end{align*}
Then the outcome set would be,
\begin{align*}
A = \{f_1, f_2, f_3\}
\end{align*}
\end{enumerate}
\end{frame}

\begin{frame}{Solution}
\begin{enumerate}
\item[(ii)] Given condition,
\begin{align*}
\{x \leq 5\}
\end{align*}
This can be written as,
\begin{align*}
\x{f_i} & \leq 5\\
10i & \leq 5\\
i & \leq 0.5\\
\end{align*}
But $i$ can only take the values from $1$ to $6$.
So the outcome set would be empty.
\begin{align*}
B = \phi = \{\} 
\end{align*}
\end{enumerate}
\end{frame}

\begin{frame}{Solution}
\begin{enumerate}
\item[(iii)] Given condition,
\begin{align*}
\{20 \leq x \leq 35\}
\end{align*}
This can be written as,
\begin{align*}
20 \leq \x{f_i} & \leq 35\\
20 \leq 10i & \leq 35\\
2 \leq i & \leq 3.5\\
\end{align*}
So the outcome set would be,
\begin{align*}
C = \{f_2, f_3\} 
\end{align*}
\end{enumerate}
\end{frame}

\begin{frame}{Solution}
\begin{enumerate}
\item[(iv)] Given condition,
\begin{align*}
\{x = 40\}
\end{align*}
This can be written as,
\begin{align*}
\x{f_i} & = 40\\
10i & = 40\\
i & = 4\\
\end{align*}
So the outcome set would be,
\begin{align*}
D = \{f_4\} 
\end{align*}
\end{enumerate}
\end{frame}

\begin{frame}{Solution}
\begin{enumerate}
\item[(v)] Given condition,
\begin{align*}
\{x = 5\}
\end{align*}
This can be written as,
\begin{align*}
\x{f_i} & = 5\\
10i & = 5\\
i & = 0.5\\
\end{align*}
But $i$ can only take the values from $1$ to $6$.
So the outcome set would be empty.
\begin{align*}
E = \phi = \{\} 
\end{align*}
\end{enumerate}
\end{frame}

\end{document}\textsl{•}