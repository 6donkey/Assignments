\documentclass{beamer}
\usepackage{changepage}
\usepackage{amsmath,amsfonts,mathtools}


\providecommand{\x}[1]{\ensuremath{\text{x}\left(#1\right)}}
\providecommand{\p}[1]{\ensuremath{{P_X}\left(#1\right)}}
\providecommand{\f}[1]{\ensuremath{{F_X}\left(#1\right)}}

% Theme choice:
\usetheme{Singapore}

% Title page details: 
\title{Assignment-7\\(Papoulis Chapter 8 Example 8.27)} 
\author{J Sai Sri Hari Vamshi\\ AI21BTECH11014}
\date{\today}
\logo{\large \LaTeX{}}


\begin{document}

% Title page frame
\begin{frame}
    \titlepage 
\end{frame}

% Remove logo from the next slides
%\logo{}


% Outline frame
\begin{frame}{Contents}
    \tableofcontents
\end{frame}

\section{Question}
\begin{frame}{Question}
	Suppose that $f(x, \theta) ~ \theta e^{-\theta x} U(x)$. Then test the likelihood ratio hypothesis \\
	\begin{align*}
	H_0:0<\theta\leq\theta_0 \ \ against \ \ H_1:\theta>\theta_0
	\end{align*}
\end{frame}

\section{Solution}

\begin{frame}{Solution}
In this problem, $\Phi_0$ is the segment $0 < \theta \leq \theta_0 $ of the real line and $\Phi$ is the half-line
    $\theta < 0$. Thus both hypotheses are composite. The likelihood function
    \begin{align}
        f(X,\theta) = \theta^n e^{-n\bar{x}\theta} 
    \end{align}
    is shown for $\bar{x} > 1/\theta_0$ and $\bar{x} < 1/\theta_0$. In the half-line $\theta_0 > 0$ this function
    is maximum for $\theta = 1/\bar{x}$. In the interval $0 < \theta \leq \theta_0$ it is maximum for $\theta = 1/\bar{x}$ if
    $\bar{x}> 1/\theta_0$ and for $\theta = \theta_0$ if $\bar{x} < 1/\theta_0$. Hence
    \begin{align}
      \theta_m = \frac{1}{x} ~~~~~ \theta_{m0}= \begin{cases}
        1/\bar{x} ~~for ~~\bar{x} > 1/\theta_0\\
        \theta_0 ~~for ~~\bar{x} < 1/\theta_0
      \end{cases}
    \end{align}
\end{frame}

\begin{frame}{Solution}
    The likelihood ratio equals
    \begin{align}
      \lambda = \begin{cases}
        1 ~~for ~~\bar{x} > 1/\theta_0\\
        (\bar{x}\theta_0)^ne^{-n\theta_0\bar{x}+n\theta_0}~~for ~~\bar{x} < 1/\theta_0
      \end{cases}
    \end{align}
    We reject $H_o$ if $\lambda < c$ or, equivalently, if $\bar{x} < c_1$, where $c_1$ equals the $\alpha$ percentile of the
    random variable $\bar{x}$. 
\end{frame}

\end{document}