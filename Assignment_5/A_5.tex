\documentclass{beamer}
\usepackage{changepage}


\providecommand{\x}[1]{\ensuremath{\text{X}\left(#1\right)}}
\providecommand{\p}[1]{\ensuremath{{P_X}\left(#1\right)}}
\providecommand{\f}[1]{\ensuremath{{F_X}\left(#1\right)}}

% Theme choice:
\usetheme{Singapore}

% Title page details: 
\title{Assignment-5\\(Papoulis Chapter 2 Example 2.2)} 
\author{J Sai Sri Hari Vamshi\\ AI21BTECH11014}
\date{\today}
\logo{\large \LaTeX{}}


\begin{document}

% Title page frame
\begin{frame}
    \titlepage 
\end{frame}

% Remove logo from the next slides
%\logo{}


% Outline frame
\begin{frame}{Contents}
    \tableofcontents
\end{frame}

\section{Question}
\begin{frame}{Question}
	Suppose a coin is tossed twice and the outcomes are recorded. Then:
		\begin{enumerate}[(i)]
		\item What is the sample space of the outcomes obtained?
		\item What is the set of outcomes with heads at the first toss?
		\item What is the set of outcomes with only one heads?
		\item What is the set of outcomes with heads showing atleast once?
	\end{enumerate}
\end{frame}

\section{Solution}
\begin{frame}{Solution}
When a coin is tossed twice
	\begin{enumerate}[(i)]
		\item The sample space contains the elements:
		\begin{align*}
			S = \{hh,ht,th,tt\}
		\end{align*}
		the set $S$ has $2^4 = 16$ subsets since it has $4$ elements in its sample space.
		\item The set of outcomes with heads at the first toss:
		\begin{align*}
			A = \{heads\ at\ the\ first\ toss\} = \{hh,ht\}
		\end{align*}
	\end{enumerate}
\end{frame}
\begin{frame}{Solution}
	\begin{enumerate}
	\item[(iii)] The set of outcomes with only one heads:
		\begin{align*}
			B = \{only\ one\ heads\} = \{ht,th\}
		\end{align*}
	\item[(iv)] The set of outcomes with heads showing atleast once:
		\begin{align*}
			C = \{heads\ shows\ atleast\ once\} = \{hh,ht,th\}
		\end{align*}
	\end{enumerate}

\end{frame}

\end{document}\textsl{•}