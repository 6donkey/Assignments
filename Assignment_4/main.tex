\documentclass[journal,12pt,twocolumn]{article}\usepackage[margin=1.25 in]{geometry}
\usepackage{amsmath, amssymb}
\usepackage{multicol}
\usepackage{graphicx}
\usepackage[shortlabels]{enumitem}

\providecommand{\x}[1]{\ensuremath{\text{X}\left(#1\right)}}

\addtolength{\topmargin}{-.4 in}

\title{\LARGE{\textbf{Assignment-3}\\(CBSE 11th Ex 16.3)}}
\author{\normalsize J Sai Sri Hari Vamshi\\ \footnotesize AI21BTECH11014}
\date{}

\begin{document}
\maketitle
\begin{center}
    \textbf{\large example 23:}
\end{center}

\noindent A bag contains $2$ white and $1$ red balls. One ball is drawn at random and then put back into the box after noting its colour. The process is repeated again. If X denotes the number of red balls recorded in the two draws, describe X.

\begin{center}
    \textbf{\large Solution:}
\end{center}

\noindent Let the balls in the bag be denoted bu $w_1, w_2$ and $r$ as the two white balls are not identical.
\noindent Then the sample space is:
	\begin{align*}
	S & = \{w_1 w_1,\ w_1 w_2,\ w_2 w_2,\ w_2 w_1,\ w_1 r,\ w_2 r,\ r w_1,\ r 	w_2,\ r r\}
	\end{align*}
\noindent Let $\omega$ be an element of the sample space. i.e.,
	\begin{align*}
	\omega & \in S
	\end{align*}
\noindent Given that X denotes the number of red balls, then 
	\begin{align*}
	\x{\omega} = \text{No. of red balls in }\omega
	\end{align*}
\noindent Therefore,
	\begin{align*}
	\x{\{w_1 w_1\}} = \x{\{w_1 w_2\}} = \x{\{w_2 w_1\}} = \x{\{w_2 w_2\}} & = 0\\
	\x{\{r\ w_1\}} = \x{\{r\ w_2\}} = \x{\{w_1\ r\}} = \x{\{w_2\ r\}} & = 1\\
	\x{\{r\ r\}} & = 2
	\end{align*}
\noindent Thus X is a random variable with values $0,\ 1$ and $2$.
\end{document}